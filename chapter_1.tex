\chapter{Introduction} \label{chapter1}
As early as 3350BC, humans utilized wheels to ease their lives. For more than
five thousand years, wheels were used for many things in many ways, but it
wasn't until 1817 when Karl von Drais invented the ``running machine'': the
first single-track two-wheeled balancing vehicle. Developments to the ``running
machine'', notably pneumatic tires (1845), pedalled drive trains (1866), steel
spoked wheels (1868), two-speed hub gear (1896), and butted frame tubes (1897)
resulted in what we would comfortably call a bicycle~\cite{Wilson2004}. The
technology of the bicycle has been under constant development, but by the early
1900's the form of the bicycle had stabilized to a large degree. A mathematical
framework for the motion of bicycle began around this same
time~\cite{Whipple1899}. Since Whipple's work, hundreds of authors have approach
the analysis of bicycles and motorcycles, with many different goals: some to
simply understand how this amazing machine can balance, others to understand
how humans are able to control bicycles, and still others to understand how the
design of the vehicle can be adjusted to meet some desired requirements.

It took more than 100 years for the academic community to come to agreement on
how the model presented by Whipple in 1899 should behave. This model makes a
large number of simplifying assumptions to render the analysis tractable.
Models which include many more degrees of freedom or have less restrictive
modelling assumptions are much less well understood. The benchmark paper by
Meijaard~\cite{Meijaard2007} went a long way to reducing the fallacies and
misinformation about how bicycles behave. The same type of benchmarking needs
to be applied to more complicated models before there can be any hope of having
an intelligent discussion of their dynamics.

The first goal of this dissertation is to improve the understanding
understanding of mechanistic aspects of bicycle modelling, in particular the
Whipple model. \hyperref[chapter2]{Chapter 2} presents the derivation of nonlinear
equations of motion for the Whipple bicycle model which have been verified to
give identical results as those results presented by well established
researchers in the field. While the results are equivalent to previous work,
the derivation utilizes a unique choice of parameters which offer several
advantages over other common choices.  In particular, fewer parameters are need
to describe the same class of bicycles, the parameters are not coupled as are
the common choice of parameters, and the parameters exploit the front/rear
geometric symmetry of the bicycle to streamline the derivation and clarify the
resulting dynamic equations.

The second goal of this dissertation is to present a clear and explicit
treatment of the linearization of the nonlinear equations of motion for systems
with configuration and velocity constraints. The bicycle is such a system, and
the work of \hyperref[chapter3]{Chapter 3} generalizes techniques which were
initially developed to linearize the bicycle dynamic equations. The techniques
presented in \hyperref[chapter3]{Chapter 3} address common tasks such as
stability analysis and formulation of correct linearized state space equations
for use in control system design. All of the techniques presented in
\hyperref[chapter3]{Chapter 3} have been applied to the Whipple bicycle model
and the results have been validated numerically.

Finally, the modelling techniques of \hyperref[chapter2]{Chapter 2} and the
linearization techniques of \hyperref[chapter3]{Chapter 3} are applied to the
non trivial task of building a robot bicycle which permits experimental
validation of the Whipple model. Only a handful of studies to date have
performed rigorous experimental validation of the Whipple model. In particular,
none of the studies to date tackle the issue of experimentally validating the
dynamic behavior outside the stable speed range. The real time control system
(implemented using the techniques of \hyperref[chapter2]{Chapter 2} and
\hyperref[chapter3]{Chapter 3}) permits exactly this.

It is my sincere hope that is work is useful to students and researchers of
dynamics, control, and those interested in two-wheeled vehicles. The bicycle
has provided me with a fertile playground for all of these topics and I have no
doubt it will do the same for many others for many years to come.
