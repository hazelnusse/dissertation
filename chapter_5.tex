\chapter{Conclusions and suggestions for future work}
The nonlinear equations of motion for the bicycle model described in
\hyperref[chapter2]{Chapter 2} were formulated using bicycle gyrostat
parameters and subsequently linearized using the linearization procedure
described in \hyperref[chapter3]{Chapter 3}. These linearized equations were
found to match previously published eigenvalues for a set of benchmark
parameters, thereby establishing that no mistakes were made in the derivation.
However, this does not establish the soundness of the modelling assumptions.
The use of the linearized dynamic equations to design a control system that
balances the bicycle both in simulation and in practice does however, to some
degree, establish that the Whipple bicycle model is at least descriptive for a
control system based upon its assumptions to keep an unmanned bicycle from
falling over. To what degree the Whipple model is accurate and how exactly to
quantify the degree to which it is accurate was not concretely established by
this work.

There are several areas where this issue can be addressed. A more careful
measurement of all of the physical parameters of the bicycle would be an
inexpensive way to improve the knowledge of the assumed plant. At the some
time, issues such as inertial asymmetries of the four rigid bodies in the real
bicycle should be either eliminated or, if it is not possible to eliminate
entirely, quantified. This may necessitate the need to reformulate the model of
\hyperref[chapter2]{Chapter 2} to include inertial asymmetries. A change to
the model such as allowing for the mass center of the frame and fork to lie
outside the plane of the wheel, would be a simple addition, and similar such
modifications could be added if necessary. Other improvements in the design of the
state estimator (i.e., a reduced order observer) would also be worth testing to
see if they improve the performance characteristics of the controller. A final
high value, low cost, would be to use a higher resolution optical encoder (a
drop in replacement encoder exists that would yield 20000 quadrature counts per
wheel revolution). This improvement would to reduce discretization jitter in
the speed measurement which in turn would reduce jitter in the gain scheduling
lookup. It is unclear whether this concern is actually justified, but
nevertheless, it is a simple and cheap fix.

The goal of applying additive sinusoidal disturbance steer torques as a means
to excite specific frequencies did not work as well as planned. One problem was
that a zero mean disturbance steer torque caused a non-zero mean yaw rate,
despite the yaw rate command of zero. This may be addressable by refinement of
the control system and how the disturbance signal is applied.

Commanding circular motions (i.e., $\dot{\psi}_c \ne 0$) of the robotic bicycle
was not attempted due to lack of space and time. The theory of steady turning
bicycles is less developed than that of bicycles travelling in a straight line,
so there is substantial room for investigations of steady turns, both in terms
of numerical studies of the model, as well as experimental validations of the
model in operating conditions other than upright steady forward cruise.

With the improvements mentioned, more rigorous system identification
experiments would be possible and the validity of the inadequacies of the
Whipple model could be more precisely quantified. It is my suspicion that the
lack of a tire model is likely the first place to look when making improvements
to the model, though there may be other simple additions to the model, such as
using the torus model of the wheel instead of the knife edged model that could
potentially improve model fidelity. Simple models of tires do exist and can be
be added to the bicycle model easily to determine how far a simple tire model
can be taken. If more sophisticated tire models prove necessary, measurement of
tire viscoelastic properties would required.

This dissertation, as with any work, is never fully ``finished''. The \LaTeX{}
source code, scripts, images, and data used for generating figures is available
online~\cite{Peterson2013}.

