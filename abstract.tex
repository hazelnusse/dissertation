This dissertation explores bicycle dynamics through an extended Whipple bicycle
model, formulation of nonlinear equations of motion of this model, subsequent
linearization of the nonlinear equations of motion about arbitrary operating
conditions, and comparisons between model predictions and experimental
observations performed on a robotic bicycle.


%explores the effects of a compliant (i.e.,
%visco-elastic) ground-foot contact on simple, passive dynamic walking
%(PDW) models. Passive dynamic walkers are uncontrolled mechanical systems
%that can move down shallow slopes with a human-like gait. The motivation
%for this work stems from the physical and emotional challenges faced by
%lower extremity amputees; however, many of the results are also applicable
%to bipedal robots. A numerical simulation approach was chosen because
%biomechanical tests of human subjects often yield inconclusive or
%contradictory results. Analysis of the PDW models focused on stability and
%clinically relevant gait measures, such as speed, step length, and
%energetic efficiency. Stability was quantified using the reciprocal of the
%gait sensitivity norm (rGSN).
%
%Chapter 2 compared a compliant PDW model to two non-compliant models. The
%results showed that foot compliance decreased the range of ground slopes
%capable of sustaining stable period-one gaits, but improved the peak rGSN.
%Step length and average forward speed as a function of ground slope was
%also largest in the compliant model.
%
%Chapter 3 explored the effects of visco-elastic parameter variations. In
%contrast to previously reported results, the non-linear compliant feet
%limited the ability to make generalizations regarding the effects of a
%parameter on the walker's performance. The rGSN was most affected by
%variations of the spring's deformation exponent but was also strongly
%affected by the other four visco-elastic parameters.
%
%Chapter 4 sought to optimize the visco-elastic parameters in terms of the
%rGSN. The optimal parameter set was then compared to seven commercially
%available prosthetic feet and a rigid model. The search on a subset of
%physically realizable visco-elastic parameters found a solution with a
%rGSN 17\% larger than the best commercially available prosthetic foot and
%61\% larger than a rigid footed model. Expanding the search domain would
%likely yield even better results. Further, all of the compliant feet
%increased energetic efficiency relative to the rigid model; however, the
%optimal parameter set provided the least benefit. Assuming that this
%optimization criterion is correlated with amputee walking, these results
%suggest that commercially available prosthetic feet should be redesigned.

