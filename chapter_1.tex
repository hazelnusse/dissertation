\chapter{Introduction}
%\setcounter{section}{-1}
\section{Motivation}

As early as 3350BC, homo sapiens utilized wheels to ease
their lives \needscite{OldestWheel}. For more than five thousand years, wheels were
used for many things in many ways, but it wasn't until 1817 when Karl von Drais
invented the ``running machine'': the first single-track two-wheeled balancing
vehicle. Developments to the ``running machine'', notably pnuematic tires (1845),
pedalled drivetrains (1866), steel spoked wheels (1868), two-speed hub gear
(1896), and butted tubes (1897) resulted in what we would comfortably call a
bicycle. The technology of the bicycle has been under constant development, but
by the early 1900's the form of the bicycle had stabilized to a large degree:
pnuematic tires, spoked wheels, hand or foot brakes, and a drivetrain with one
or more gears. A mathematical framework for the motion of bicycle began
around this same time\cite{Whipple1899}.
