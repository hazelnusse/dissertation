\chapter{Mathematical model of bicycle motion} \label{chapter2}

\section{Introduction}

\section{Physical parameters}

\subsection{Parameter conversion}\label{model:parameter_conversion}
We make a slight addition to the physical parameters presented
in~\cite{Meijaard2007} by modelling the wheels as torii instead of thin disks.
This results in two extra parameters: the tori minor radii $t_R$ and $t_F$ of
the rear and front wheels respectively. This brings the number of parameters
from 25 to 27. Converting from these 27 parameters to the 23 gyrostat
parameters is always possible. However, the opposite is not generally true
because the mapping is not injective (it is one-to-many). In all of the
equations that follow, the symbols used for the parameters presented
in~\cite{Meijaard2007} are on the right side of the equality, while the symbols
used to describe the gyrostat parameters are on the left of the equality. Some
parameters have very similar symbols but should not be confused as being the
same (i.e., $m_R \ne m_r$).

Of the twenty three gyrostat parameters, six have identical counterparts in the Meijaard
paramters
\begin{align}
  J_r &= I_{Ryy} \\
  R_r &= r_R \\
  r_r &= t_R\\
  J_f &= I_{Fyy} \\
  R_f &= r_F \\
  r_f &= t_F
\end{align}
while two are trivially related
\begin{align}
    m_r &= m_R + m_B \\
    m_f &= m_F + m_H
\end{align}
The remaining fifteen (23-6-2=15) gyrostat parameters are related to the
Meijaard parameters as follows






\section{Kinematics}

\section{Dynamics}

