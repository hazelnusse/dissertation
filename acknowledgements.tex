\label{acknowledgements}
This work would not have been possible without the support of many people.
First and foremost, I would like thank my parents, John and Lyn. They have been
there for me at every stage of my life and I am thankful for their support,
love, care, and for inspiring me to be inquisitive about the world around me.
Next, I would like to thank my fianc\'ee Cassandra Ann Paul, who has been
essential to this dissertation. Cassandra has encouraged me when I needed it
most, chastized me when I deserved it, and kept me positive throughout a very
stressful final year of graduate school. Cassandra is a reminder of what is
important in life, and I look forward to building our lives together.

I have had many excellent teachers in my life and this dissertation is a
reflection of them me. I would to thank Joan Owen, Joyce Hodgkinson, and
Elizabeth Terwilliger for their thoughtful and forward thinking work during my
early years at Tam Creek School. They planted the seeds of civic duty and
environmental stewardship in many young children and I am thankful to have been
one of their students. I would like to thank my high school chemistry teacher,
Nina Tychinin, for encouraging me to explore my interests in science. Near the
end of my undergraduate education, after returning from a two year activation
with the United States Marine Corps, I enrolled in an applied dynamics course
with Dr. Steven Shaw and a control systems course with Dr. Jo\~{a}o P.
Hespanha. These two teachers are directly responsible for igniting my
fascination with dynamics and control, and I would like to thank them for
careful and inspired presentation of these two beautiful subjects.

I would like to thank my Ph.D. advisor Dr. Mont Hubbard for his patience and
clarity of thought, and for the many discussions about dynamics, control, and
bicycles. It has been a true pleasure to work with Mont and I am grateful to
have learned so much from him; I hope I was as good a student to him as he was
an advisor to me. I would like to thank Dr. Ron Hess for his support and
feedback throughout the period of the NSF grant and specifically for his help
with design of the control system of the robotic bicycle. I would like to thank
Dr. Fidelis Eke for his extremely clear presentation of multibody dynamics and
Kane's method in particular. I would like to thank Dr. James
Crutchfield for his inspirational course on nonlinear dynamics in which I was
first exposed to the Python programming language.

The work of Dr. Arend L. Schwab, Dr. Jeremy Papadopoulos, Dr. Andy Ruina, and
Dr. Jaap Meijaard has had a large impact on this work. Their clarity of
presentation and the many thoughtful discussions we have had about bicycle
dynamics and control have significantly improved my understanding of bicycles
and dynamics. I would like to extend special thanks to Dr. Arend Scwab and Dr.
Jeremy Papadopoulos for the many in-depth discussions and emails we have had.
It is my hope that the work in this dissertation is an important addition to
the groundwork that you and many others have laid.

There are several students I would like to thank. First and foremost, I would
like to thank my friend Jason Moore. Jason has enriched my life both inside and
outside the walls of the academy (which we are both trying to tear down). I
have enjoyed discussing bicycles, religion, computers, math, and philosophy
with Jason and I credit him with keeping me on my toes and testing my views on
the world. Thomas Johnston was a great companion in graduate school, and I am
thankful for the many hours of discussion of dynamics we have had. I would like
to thank all of the members, past and present, of the Sports Biomechanics
Laboratory.

I would like to thank Ond\v{r}ej \v{C}ert\'{i}k for his substantial commitment
to symbolic mathematics with the SymPy project and for helping me build a tool
with which to study dynamics. Ond\v{r}ej has demonstrated excellent leadership
and has created a wonderful community for symbolic mathematics.  His
selflessness and commitment have helped to make my dissertation possible.

In the summer of 2011 Gilbert Gede and I worked closely to add classical
mechanics functionality to the SymPy project. It was also during this time that
the ideas for Chapter 3 of this dissertation were planted and initially
developed. Gilbert's hard work and collaboration on the development of these
ideas is greatly appreciated.

A large number of people contributed to the success of the robot bicycle.
Armando Lee provided crucial welded several parts and operated the wire
electronic discharge machine for several parts of the robotic bicycle. Benny
Brown provided hours of much need electrical engineering experience and know
how. Kenny Koller helped me understand embedded systems and provoked a lot of
thought about careful design of software which interacts directly with hardware
resources. Kenneth Lyons helped with steer angle calibration, and Derek Pell
helped fabricate a number of components used for measuring the physical
parameters of the bicycle. I appreciate all of your help.

Oliver Z. Lee deserves special recognition for his help with the robot bicycle.
Oliver and I worked shoulder to shoulder for the first 7 months of 2013 and we
came to know each other quite well. Oliver's contribution to this dissertation
was significant and essential; in particular, the robot bicycle would not have
balanced had it not been for his hard work, focus, and impressive skills in
software and control systems.

Finally, I would like to thank Martin Hansen, Tony Merz, Ron Blinn, Erik
Gambera, Allen Donhauser, Red Ross, Dan Novak, Dan Freudenberger, and Steve
Wyrostok from Sunshine Bicycle Center for their support and the many
interesting discussions about bicycle we had. My fascination with bicycles was
kindled by these discussions.

This material is partially based upon work supported by the National Science
Foundation under Grant No. 0928339. Any opinions, findings, and conclusions or
recommendations expressed in this material are those of the author(s) and do
not necessarily reflect the views of the National Science Foundation.

